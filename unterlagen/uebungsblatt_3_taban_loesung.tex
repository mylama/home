\documentclass[a4paper,12pt]{report}

\usepackage{geometry}
\geometry{a4paper,left=18mm,right=18mm, top=2cm, bottom=2cm}

\usepackage{lmodern}
\usepackage[T1]{fontenc}
\usepackage{eurosym}
\usepackage{setspace}
\usepackage[utf8]{inputenc}
\usepackage{graphicx}
\usepackage[ngerman]{babel}
\usepackage[solution_off]{srdp-mathematik} % solution_on/off
\setcounter{Zufall}{0}
\usepackage{listings}
\usepackage{fancyhdr}
\renewcommand{\headrulewidth}{0pt}

\pagestyle{fancy} %PAGESTYLE: empty, plain, fancy
\onehalfspacing %Zeilenabstand
\setcounter{secnumdepth}{-1} % keine Nummerierung der Ueberschriften


%
%
%%%%%%%%%%%%%%%%%%%%%%%%%%%%%%%%%%%%%%%%%%%%%%%%%%%%%%%%%%%%%%%%%%%%%%%%%%%%%%%%%%%%%%%%%% DOKUMENT - ANFANG %%%%%%%%%%%%%%%%%%%%%%%%%%%%%%%%%%%%%%%%%%%%%%%%%%%%%%%%%%%%%%%%%%%%%%%%%%%%%%%%%%%%%%%
%
%
\begin{document}
\begin{center}
\begin{large}
Arbeitsblatt - Tabellarische Anordnung
\end{large}
\end{center}
\begin{enumerate}
\item align:\\
\begin{align*}
4x-2y+3z&=5\\
3x&=-2y+4z-7\\
3x+2y&=4z+7\\
3x-2y+4z+10&=0\\
\end{align*}


\item cases:\\
$$f(x)=\begin{cases}
3x-2&x<2\\
4x^2+\frac{3}{x}&x=2\\
-2^x+3&2<x<4\\
4x+5&x\geq 4\end{cases}$$


\item Tabelle:\\
\begin{center}
\begin{tabular}{|c|c|}\hline
	\cellcolor[gray]{0.9}Jahr&\cellcolor[gray]{0.9}Baukostenindex\\ \hline
	2010&$+3,2\,\%$\\ \hline
	2011&$+2,3\,\%$\\ \hline
	2012&$+2,1\,\%$\\ \hline
	2013&$+1,9\,\%$\\ \hline
	2014&$+1,1\,\%$\\ \hline
	\end{tabular}\end{center}
	
	\textit{Tipp:} Das Einfärben von Zellen funktioniert mit dem Befehl: \textbackslash cellcolor[gray]\{0.9\}
	
\item Tabelle 2:\\
Kreuze diejenige(n) Zahl(en) an, die aus der Zahlenmenge $\mathbb{Z}$ ist/sind!

\begin{center}
\begin{tabular}{|c|c|}\hline
$\frac{25}{5}$&\Square \\ \hline
$-\sqrt[3]{8}$&\Square \\ \hline
$0,\overline{4}$&\Square \\ \hline
$1,4\cdot 10^{-3}$&\Square \\ \hline
$-1,4\cdot 10^{3}$&\Square \\ \hline
\end{tabular}
\end{center}

\textit{Tipp:} Die leeren Kästchen erhält man durch den Befehl: \textbackslash Square

\end{enumerate}

\fancyfoot[c]{\begin{footnotesize}Christoph Weberndorfer, Matthias Konzett\\
lama.helpme@gmail.com\end{footnotesize}}	

\end{document}