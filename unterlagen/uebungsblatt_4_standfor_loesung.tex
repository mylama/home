\documentclass[a4paper,12pt]{report}

\usepackage{geometry}
\geometry{a4paper,left=18mm,right=18mm, top=2cm, bottom=2cm} 

\usepackage{lmodern}
\usepackage[T1]{fontenc}
\usepackage[utf8]{inputenc}
\usepackage[ngerman]{babel}
\usepackage[solution_off]{srdp-mathematik} % solution_on/off
\setcounter{Zufall}{0}
\usepackage{listings}
\usepackage{fancyhdr}
\renewcommand{\headrulewidth}{0pt}

\pagestyle{plain} %PAGESTYLE: empty, plain, fancy
\onehalfspacing %Zeilenabstand
\setcounter{secnumdepth}{-1} % keine Nummerierung der Ueberschriften


%
%
%%%%%%%%%%%%%%%%%%%%%%%%%%%%%%%%%%%%%%%%%%%%%%%%%%%%%%%%%%%%%%%%%%%%%%%%%%%%%%%%%%%%%%%%%% DOKUMENT - ANFANG %%%%%%%%%%%%%%%%%%%%%%%%%%%%%%%%%%%%%%%%%%%%%%%%%%%%%%%%%%%%%%%%%%%%%%%%%%%%%%%%%%%%%%%
%
%
\begin{document}
\setcounter{Antworten}{1}
\section{Übungsblatt Typ-1 Formate}

\begin{beispiel}[AG 1.1]{1} %PUNKTE DES BEISPIELS
Die Menge $M = \{ x \in \mathbb{Q}\,|\,2 < x < 5\}$ ist eine Teilmenge der rationalen Zahlen.

Kreuze die beiden zutreffenden Aussagen an.

\multiplechoice[5]{  %Anzahl der Antwortmoeglichkeiten, Standard: 5
				L1={4,99 ist die größte Zahl, die zur Menge $M$ gehört.},   %1. Antwortmoeglichkeit 
				L2={Es gibt unendlich viele Zahlen in der Menge $M$, die kleiner als 2,1 sind.},   %2. Antwortmoeglichkeit
				L3={Jede reelle Zahl, die größer als 2 und kleiner als 5 ist, ist in der Menge $M$
enthalten.},   %3. Antwortmoeglichkeit
				L4={Alle Elemente der Menge $M$ können in der Form $\frac{a}{b}$
geschrieben werden, wobei $a$ und $b$ ganze Zahlen sind und $b \neq 0$ ist.},   %4. Antwortmoeglichkeit
				L5={Die Menge $M$ enthält keine Zahlen aus der Menge der komplexen Zahlen.},	 %5. Antwortmoeglichkeit
				L6={},	 %6. Antwortmoeglichkeit
				L7={},	 %7. Antwortmoeglichkeit
				L8={},	 %8. Antwortmoeglichkeit
				L9={},	 %9. Antwortmoeglichkeit
				%% LOESUNG: %%
				A1=2,  % 1. Antwort
				A2=4,	 % 2. Antwort
				A3=0,  % 3. Antwort
				A4=0,  % 4. Antwort
				A5=0,  % 5. Antwort
				}
\end{beispiel}

\begin{beispiel}[K7 - KKK]{1} %PUNKTE DES BEISPIELS
Gegeben sind mehrere Gleichungen.

Kreuze jene Gleichung an, die eine Kugel beschreibt.

\langmultiplechoice[6]{  %Anzahl der Antwortmoeglichkeiten, Standard: 5
				L1={$x^2+(y-2)^2-(z-4)^2=16$},   %1. Antwortmoeglichkeit 
				L2={$(x-4)^3+(y-1)^3=64$},   %2. Antwortmoeglichkeit
				L3={$(x-1)^2+(y+3)^2+z^2=5$},   %3. Antwortmoeglichkeit
				L4={$x\cdot(y-2)^2\cdot(z-1)^2=9$},   %4. Antwortmoeglichkeit
				L5={$x^2+y^2+z^2-4x+2z+10=0$},	 %5. Antwortmoeglichkeit
				L6={$3x^2+(y-2)^2+2z^2=16$},	 %6. Antwortmoeglichkeit
				L7={},	 %7. Antwortmoeglichkeit
				L8={},	 %8. Antwortmoeglichkeit
				L9={},	 %9. Antwortmoeglichkeit
				%% LOESUNG: %%
				A1=3,  % 1. Antwort
				A2=0,	 % 2. Antwort
				A3=0,  % 3. Antwort
				A4=0,  % 4. Antwort
				A5=0,  % 5. Antwort
				}
				\end{beispiel}
				
				\begin{beispiel} [AG 1.2]{1} %PUNKTE DES BEISPIELS
Ordne jeder Aussage den richtigen Term zu!
\zuordnen[0.25]{
				R1={Der Energieverbrauch E ist um 10\,\% gestiegen.},				% Response 1
				R2={Der Energieverbrauch E ist auf das Doppelte gestiegen.},				% Response 2
				R3={Der Energieverbrauch E wurde um 10\,\% gesenkt.},				% Response 3
				R4={Der Energieverbrauch E ist um das Doppelte gestiegen.},				% Response 4
				%% Moegliche Zuordnungen: %%
				A={$E+E$}, 				%Moeglichkeit A  
				B={$E+0,10$}, 				%Moeglichkeit B  
				C={$E:1,1$}, 				%Moeglichkeit C  
				D={$E\cdot1,1$}, 				%Moeglichkeit D  
				E={$E\cdot0,9$}, 				%Moeglichkeit E  
				F={$3\cdot E$}, 				%Moeglichkeit F  
				%% LOESUNG: %%
				A1={D},				% 1. richtige Zuordnung
				A2={A},				% 2. richtige Zuordnung
				A3={E},				% 3. richtige Zuordnung
				A4={F},				% 4. richtige Zuordnung
				}
\end{beispiel}

\begin{beispiel}[FA 6.4]{1}
Gegeben sind vier Funktionen $f_1, f_2, f_3$ und $f_4$ mit den nachfolgend angeführten Funktionsgleichungen.

Ordne den Funktionen jeweils die zugehörige Periodenlänge $p$ zu.

\zuordnen{
				R1={$f_1(x)=3\cdot\sin(2\cdot x)$},				% Response 1
				R2={$f_2(x)=\pi\cdot\sin(4\pi\cdot x)$},				% Response 2
				R3={$f_3(x)=\sin(0,5\pi\cdot x)$},				% Response 3
				R4={$f_4(x)=4\cdot\sin(x)$},				% Response 4
				%% Moegliche Zuordnungen: %%
				A={$p=4$}, 				%Moeglichkeit A  
				B={$p=3\pi$}, 				%Moeglichkeit B  
				C={$p=\pi$}, 				%Moeglichkeit C  
				D={$p=2\pi$}, 				%Moeglichkeit D  
				E={$p=0,5$}, 				%Moeglichkeit E  
				F={$p=3$}, 				%Moeglichkeit F  
				%% LOESUNG: %%
				A1={C},				% 1. richtige Zuordnung
				A2={E},				% 2. richtige Zuordnung
				A3={A},				% 3. richtige Zuordnung
				A4={D},				% 4. richtige Zuordnung
				}
\end{beispiel}

\begin{beispiel}[AG 2.3]{1} %PUNKTE DES BEISPIELS
				Die Anzahl der Lösungen der quadratischen Gleichung $rx^2+sx+t=0$ in der Menge der reellen Zahlen hängt von den Koeffizienten $r,s$ und $t$ ab.
				
				\lueckentext{
								text={Die quadratische Gleichung $rx^2+sx+t=0$ hat genau dann \textbf{für alle} \mbox{$r\neq 0;r,s,t\in\mathbb{R}$} \gap , wenn \gap gilt.}, 	%Lueckentext Luecke=\gap
								L1={zwei reelle Lösungen}, 		%1.Moeglichkeit links  
								L2={keine reelle Lösung}, 		%2.Moeglichkeit links
								L3={genau eine reelle Lösung}, 		%3.Moeglichkeit links
								R1={$r^2-4st>0$}, 		%1.Moeglichkeit rechts 
								R2={$t^2=4rs$}, 		%2.Moeglichkeit rechts
								R3={$s^2-4rt>0$}, 		%3.Moeglichkeit rechts
								%% LOESUNG: %%
								A1=1,   % Antwort links
								A2=3		% Antwort rechts 
								}

\end{beispiel}

\begin{beispiel}[AG 3.4]{1} %PUNKTE DES BEISPIELS

Gegeben sind zwei Geraden $g$ und $h$.

Die Gleichungen der Geraden lauten $g:\, X=\Vek{3}{5}{} + s \cdot \Vek{-2}{4}{}$ und\\
 \mbox{$h:\, 2\cdot x - 4 \cdot y= -14$}.

\lueckentext[-0.2]{
				text={Die Geraden $g$ und $h$ \gap, weil \gap.}, 	%Lueckentext Luecke=\gap
				L1={sind ident}, 		%1.Moeglichkeit links  
				L2={sind parallel}, 		%2.Moeglichkeit links
				L3={stehen normal aufeinander}, 		%3.Moeglichkeit links
				R1={die Richtungsvektoren der beiden Geraden $g$ und $h$ parallel sind}, 		%1.Moeglichkeit rechts 
				R2={der Punkt $P(3|5)$ auf beiden Geraden $g$ und $h$ liegt}, 		%2.Moeglichkeit rechts
				R3={der Richtungsvektor von $g$ zum Normalvektor von $h$ parallel ist.}, 		%3.Moeglichkeit rechts
				%% LOESUNG: %%
				A1=3,   % Antwort links
				A2=3		% Antwort rechts 
				}
 
\end{beispiel}

\begin{beispiel}[AG 2.1]{1} %PUNKTE DES BEISPIELS
Ein Geldbetrag K wird auf ein Sparbuch gelegt. Er wächst in $n$ Jahren bei einem effektiven Jahreszinssatz von $p\,\%$ auf $K(n)=K\cdot \left(1+\frac{p}{100}\right)^n$.

\leer

Gib eine Formel an, die es ermöglicht, aus dem aktuellen Kontostand $K(n)$ jenen des nächsten Jahres $K(n+1)$ zu errechnen!	


\antwort{$K(n+1)=K(n)\cdot\left(1+\frac{p}{100}\right)$}				
\end{beispiel}

\begin{beispiel}[AN 3.3]{1} %PUNKTE DES BEISPIELS
Für eine Polynomfunktion $f$ gilt: $W=(2\mid y_w)$ ist ein Wendepunkt von $f$ und die Tangente an den Graphen der Funktion $f$ im Wendepunkt wird durch die Gleichung $3x+y=6$ beschrieben.

Ergänze die nachfolgenden Gleichungen so, dass sie wahre Aussagen darstellen.\leer

$f'(2)=$ \antwort[\rule{3cm}{0.3pt}]{-3}\leer

$f''(2)=$ \antwort[\rule{3cm}{0.3pt}]{0}
\end{beispiel}

\newpage
\pagestyle{fancy}
\setcounter{Antworten}{0}

\begin{scriptsize}
\section{\texttt{srdp-mathematik} Befehlsübersicht}\leer

\begin{minipage}{0.5\textwidth}
\subsection{Beispiel Umgebung}
\begin{verbatim}
\begin{beispiel}{0} %PUNKTE DES BEISPIELS

\end{beispiel}
\end{verbatim}
\end{minipage}\begin{minipage}{0.5\textwidth}
\subsection{Lange Beispiel Umgebung}
\begin{verbatim}
\begin{langesbeispiel} \item[0] %PUNKTE DES BEISPIELS

\end{langesbeispiel}
\end{verbatim}
\end{minipage}\leer


\begin{minipage}{0.5\textwidth}
\subsection{Multiplechoice}
\begin{verbatim}
\multiplechoice[5]{%Anzahl der Antworten Standard: 5
				L1={},   %1. Antwortmoeglichkeit 
				L2={},   %2. Antwortmoeglichkeit
				L3={},   %3. Antwortmoeglichkeit
				L4={},   %4. Antwortmoeglichkeit
				L5={},	 %5. Antwortmoeglichkeit
				L6={},	 %6. Antwortmoeglichkeit
				L7={},	 %7. Antwortmoeglichkeit
				L8={},	 %8. Antwortmoeglichkeit
				L9={},	 %9. Antwortmoeglichkeit
				%% LOESUNG: %%
				A1=0,  % 1. Antwort
				A2=0,	 % 2. Antwort
				A3=0,  % 3. Antwort
				A4=0,  % 4. Antwort
				A5=0,  % 5. Antwort
				}
\end{verbatim}
\end{minipage}\begin{minipage}{0.5\textwidth}
\subsection{Lange Mutiplechoice}
\begin{verbatim}
\langmultiplechoice[5]{%Anzahl der Antworten Standard: 5
				L1={},   %1. Antwortmoeglichkeit 
				L2={},   %2. Antwortmoeglichkeit
				L3={},   %3. Antwortmoeglichkeit
				L4={},   %4. Antwortmoeglichkeit
				L5={},	 %5. Antwortmoeglichkeit
				L6={},	 %6. Antwortmoeglichkeit
				L7={},	 %7. Antwortmoeglichkeit
				L8={},	 %8. Antwortmoeglichkeit
				L9={},	 %9. Antwortmoeglichkeit
				%% LOESUNG: %%
				A1=0,  % 1. Antwort
				A2=0,	 % 2. Antwort
				A3=0,  % 3. Antwort
				A4=0,  % 4. Antwort
				A5=0,  % 5. Antwort
				}
\end{verbatim}
\end{minipage}\leer


\begin{minipage}[t]{0.5\textwidth}
\subsection{Lückentext}
\begin{verbatim}
\lueckentext{
				text={}, 	%Lueckentext Luecke=\gap
				L1={}, 		%1.Moeglichkeit links  
				L2={}, 		%2.Moeglichkeit links
				L3={}, 		%3.Moeglichkeit links
				R1={}, 		%1.Moeglichkeit rechts 
				R2={}, 		%2.Moeglichkeit rechts
				R3={}, 		%3.Moeglichkeit rechts
				%% LOESUNG: %%
				A1=0,   % Antwort links
				A2=0		% Antwort rechts 
				}
\end{verbatim}
\end{minipage}\begin{minipage}[t]{0.5\textwidth}
\subsection{Zuordnen}
\begin{verbatim}
\zuordnen{
				R1={},				% Response 1
				R2={},				% Response 2
				R3={},				% Response 3
				R4={},				% Response 4
				%% Moegliche Zuordnungen: %%
				A={}, 				%Moeglichkeit A  
				B={}, 				%Moeglichkeit B  
				C={}, 				%Moeglichkeit C  
				D={}, 				%Moeglichkeit D  
				E={}, 				%Moeglichkeit E  
				F={}, 				%Moeglichkeit F  
				%% LOESUNG: %%
				A1={},				% 1. richtige Zuordnung
				A2={},				% 2. richtige Zuordnung
				A3={},				% 3. richtige Zuordnung
				A4={},				% 4. richtige Zuordnung
				}
\end{verbatim}
\end{minipage}\leer

\begin{minipage}[t]{0.5\textwidth}
\subsection{Lösungen}
\begin{verbatim}
\antwort{}
\end{verbatim}
\end{minipage}\begin{minipage}[t]{0.5\textwidth}
\subsection{Notenschlüssel}
\begin{verbatim}
\notenschluessel{0.91}{0.8}{0.64}{0.5} %mit Prozentangabe

\notenschluesselop{0.91}{0.8}{0.64}{0.5} %ohne Prozentangabe

\beurteilungsraster{0.85}{0.68}{0.5}{1/3}{ % Prozentschluessel
				T1={24}, 				% Punkte im Teil 1  
				AP={4}, 				% Ausgleichspunkte aus Teil 2  
				T2={20}, 				% Punkte im Teil 2
				}

\end{verbatim}
\end{minipage}

\end{scriptsize}

\fancyhead{}

\fancyfoot[c]{\begin{footnotesize}Christoph Weberndorfer, Matthias Konzett\\
lama.helpme@gmail.com\end{footnotesize}}	
%\fancyfoot[c]{Christoph Weberndorfer und Matthias Konzett\\
%\hspace{0,5cm} c.weberndorfer@gmail.com und m.s.konzett@gmail.com}	
\end{document}