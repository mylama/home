\documentclass[a4paper,12pt]{report}

\usepackage{geometry}
\geometry{a4paper,left=18mm,right=18mm, top=2cm, bottom=2cm}

\usepackage{lmodern}
\usepackage[T1]{fontenc}
\usepackage{eurosym}
\usepackage{setspace}
\usepackage[utf8]{inputenc}
\usepackage{graphicx}
\usepackage[ngerman]{babel}
\usepackage[solution_off]{srdp-mathematik} % solution_on/off
\setcounter{Zufall}{0}
\usepackage{listings}
\usepackage{fancyhdr}
\renewcommand{\headrulewidth}{0pt}

\pagestyle{fancy} %PAGESTYLE: empty, plain, fancy
\onehalfspacing %Zeilenabstand
\setcounter{secnumdepth}{-1} % keine Nummerierung der Ueberschriften


%
%
%%%%%%%%%%%%%%%%%%%%%%%%%%%%%%%%%%%%%%%%%%%%%%%%%%%%%%%%%%%%%%%%%%%%%%%%%%%%%%%%%%%%%%%%%% DOKUMENT - ANFANG %%%%%%%%%%%%%%%%%%%%%%%%%%%%%%%%%%%%%%%%%%%%%%%%%%%%%%%%%%%%%%%%%%%%%%%%%%%%%%%%%%%%%%%
%
%
\begin{document}
\begin{center}
\begin{large}
Arbeitsblatt - Mathematische Formeln
\end{large}
\end{center}
\begin{enumerate}
\item Endliches Intervall:\\
$$[a;b]=\{x\in\mathbb{R}\mid a\leq x\leq b\}$$

\item Durchschnitt zweier Mengen:\\
$$A\cap B=\{x\in G\mid x\in A\wedge x\in B\}$$

\item "`Große Lösungsformel"':\\
$$x_{1,2}=\dfrac{-b\pm\sqrt{b^2-4ac}}{2a}$$

\item Empirische Standardabweichung:
$$s=\sqrt{\dfrac{(x_1-\bar{x})^2+(x_2-\bar{x})^2+\cdots+(x_n-\bar{x})^2}{n}}$$

\item Vektor-Winkel-Formel:
$$\cos(\alpha)=\dfrac{\vec{a}\cdot\vec{b}}{|\vec{a}|\cdot|\vec{b}|}$$

\item Konfidenzintervall:
$$h\approx\left[p-z\cdot\sqrt{\dfrac{p\cdot(1-p)}{n}};p+z\cdot\sqrt{\dfrac{p\cdot(1-p)}{n}}\right]$$

\item Ableitung von Polynomfunktionen:\\
$$f(x)=a_nx^n+a_{n-1}x^{n-1}+\cdots+a_1x+a_0 \Rightarrow f'(x)=a_n\cdot n\cdot x^{n-1}+a_{n-1}\cdot(n-1)\cdot x^{n-2}+\cdot+a_1$$

\item Normalverteilung:
$$F(x)=\dfrac{1}{\sigma\cdot\sqrt{2\pi}}\displaystyle\int^x_{-\infty}e^{-\frac{1}{2}\cdot\left(\frac{t-\mu}{\sigma}\right)^2}\,\text{d}t$$

\item Kreuzprodukt:
$$\vec{a}\times\vec{b}=\Vek{a_1}{a_2}{a_3}\times\Vek{b_1}{b_2}{b_3}=\Vek{a_2b_3-a_3b_2}{a_3b_1-a_1b_3}{a_1b_2-a_2b_1}$$

\item Goldener Schnitt:
$$\lim\limits_{n\rightarrow\infty}\dfrac{f_n+1}{f_n}=\lim\limits_{n\rightarrow\infty}\dfrac{\Phi^{n+1}}{\Phi^n}=\Phi=\dfrac{1+\sqrt{5}}{2}\approx 1,6180339887$$
\end{enumerate}

\fancyfoot[c]{\begin{footnotesize}Christoph Weberndorfer, Matthias Konzett\\ lama.helpme@gmail.com\end{footnotesize}}	

\end{document}